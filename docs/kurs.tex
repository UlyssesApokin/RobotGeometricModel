% класс документа extreport, односторонняя печать,
% чистовая версия, 14-й кегль шрифта
\documentclass[oneside, final, 14pt]{extarticle}

% кодировка документа utf-8
\usepackage[utf8]{inputenc}
% шрифт русского текста
\usepackage[T2A]{fontenc}
% языки документа -- русский и английский
\usepackage[english, russian]{babel}
% преамбула настройка полей докумета и колонтитулов
\usepackage{vmargin}
% формат листа А4
\setpapersize{A4}
% отступ левого поля -- 3 см,
%	верхнего поля -- 1 см,
%	правого поля -- 1 см,
%	нижнего поля -- 2 см,
% высота верхнего колонитула -- 0 пт,
% отступ верхнего колонтитула от верхнего края текста -- 0 мм,
% высота нижнего колонтитула -- 0 пт,
% отступ ниж.кол. от нижнего края текста (нужен для номера стр.) -- 0 мм
\setmarginsrb{3cm}{9mm}{8mm}{8mm}{0pt}{0mm}{0mm}{17mm}
% красная строка для первого абзаца 
\usepackage{indentfirst}
% перенос строк, если они залезают на поля
\sloppy
% преамбула для работы с графикой в документе
\usepackage{graphicx}
% преамбула для удобной работы с математическими формулами
\usepackage{amsmath}

\makeatletter
\renewcommand*{\ps@plain}{%
  \let\@mkboth\@gobbletwo
  \let\@oddhead\@empty
  \def\@oddfoot{%
    \reset@font
    \hfil
    \thepage
    % \hfil % removed for aligning to the right
  }%
  \let\@evenhead\@empty
  \let\@evenfoot\@oddfoot
}
\makeatother
\pagestyle{plain}
% начало документа
\begin{document}

\begin{titlepage}
  \centerline{МИНИСТЕРСТВО НАУКИ И ВЫСШЕГО ОБРАЗОВАНИЯ РФ}
  \medskip
  \centerline{ФГБОУ ВО "<Брянский государственный технический университет">}
  \vfill
  \centerline{Кафедра "<Автоматизированные технологические системы">}
  \vfill
  \vfill
  \centerline{Компьютерное моделирование мехатронных систем}
  \vfill
  \centerline{Курсовая работа по теме:}
  \medskip
  \centerline{"<Разработать математическую модель трехзвенного робота-манипулятора">}
  \vfill
  \null \hfill
  \begin{minipage}{0.52\textwidth}
    Выполнил студент гр.\,О-20-МиР-мхт-Б:
	\par
	\medskip
	Апокин~Е.\,М.\,
	\rule{5.5cm}{0.25pt}
	\par
	\medskip
	"<\rule{1.25cm}{0.25pt}\,">\rule{5cm}{0.25pt}~2023~год
	\par
	\bigskip
	Проверил доц. к.т.н.\,:
	\par
	\medskip
	Съянов~С.\,Ю.\,
	\rule{5.5cm}{0.25pt}
	\par
	\medskip
	"<\rule{1.25cm}{0.25pt}\,">\rule{5cm}{0.25pt}~2023~год
  \end{minipage}
  \vfill
  \centerline{г.~Брянск~2023 год}
\end{titlepage}
\setcounter{page}{2}

\section*{Аннотация}

Здесь располагается текст аннотации, который я ещё не придумал.

\newpage
\thispagestyle{empty}
\tableofcontents
\newpage

\section*{Введение}
\addcontentsline{toc}{section}{Введение}

Одним из важнейших разделов компьютерного моделирования мехатронных систем является разработка систем управления многозвенными роботами--манипуляторами.
Основное назначение робота--манипулятора состоит в перемещении рабочего органа в пространстве по заданной траектории \cite{krahmalev:mod_rob}.
Поэтому среди систем, входящих в состав такого робота, исполнительная имеет особое значение.
\par
Манипуляционные роботы за счет гибкой конфигурируемости способны выполнять широкий круг всевозможных задач в условиях промышленного производства без необходимости закупки дорогостоящего специализированного оборудования, что ведет к экономии средств в перспективе \cite{borisov:mod_rob}.
Это вызывает высокий спрос на подобные системы со стороны промышленности, что подтверждает актуальность проблемы.
\par
Целью курсовой работы является разработка методов построения и анализа математических моделей манипуляционных систем, связывающих обощенные координаты робота и декартовы координаты характерной точки на третьем звене ({\itshape TCP~--- Tool Center Point}).
\par
Для достижения поставленной цели необходимо решить следующие задачи.
Проанализировать методы решения прямой задачи кинематики для многозвенных робототехнических систем.
Выбрать метод, наиболее удовлетворяющий цели работы, и обосновать выбор конкретного метода.
Выбрать кинематическую схему манипуляционного робота и посредством избранного метода построить её кинематическую модель.
Проверить правильность модели графо--аналитическим способом.
Разработать приложение с CLI-интерфейсом, предоставляющее визуализацию кинематической схемы манипуляционного робота.

\newpage

\section{Аналитический обзор}

\subsection{Анализ существующих конструкций}

Промышленный робот~--- это перепрограммируемый многофункциональный манипулятор, предназначенный для осуществления различных заранее заданных перемещений материалов, деталей, инструментов или специальных приспособлений с целью выполнения различных работ \cite{fu:rob_tech}.
\par
Промышленный робот представляет собой универсальный, оснащенный компьютером манипулятор, состоящий из нескольких твердых звеньев, последовательно соединенных вращательными или поступательными сочленениями \cite{fu:rob_tech}.
Эта цепь одним концом соединена с основанием манипулятора, другой ее конец свободен и снабжен инструментом, позволяющим воздействовать на объекты манипулирования или выполнять сборочные работы \cite{fu:rob_tech}.
Движение в сочленениях манипулятора вызывает относительное перемещение его звеньев \cite{fu:rob_tech}.
Он может воздействовать на объекты, расположенные внутри его рабочего объема \cite{fu:rob_tech}.
\par
{\itshape Роботы с многозвенной рукой}.
Некоторые роботы фактически работают как человеческа рука \cite{asfal:rob_auto}.
Такие роботы называют роботами с многозвенной рукой \cite{asfal:rob_auto}.
Основание робота поворачивается наподобие человеческого торса.
Плечевой и локтевой суставы большинства таких роботов имеют по одной оси поворота, которые перпендикулярны руке и параллельны плоскости, на которой установлено основание \cite{asfal:rob_auto}.
Запястье роботов с многозвенной рукой почти всегда может сгибаться, а поворот и вращение могут отсутствовать \cite{asfal:rob_auto}.
У большинства роботов с многозвенной рукой угол поворота кисти больше, чем у человеческой руки, которая за счет локтевого и плечевого суставов способна совершать поворот на \(270^\circ\) \cite{asfal:rob_auto}. 
\par
{\itshape Роботы, работающие в полярных координатах}.
Роботы, работающие в полярных координатах выглядят совершенно иначе, чем роботы с многозвенной рукой, но отличие фактически имеется только в третьей оси (3--ей степени подвижности) \cite{asfal:rob_auto}.
Вместо локтевого сустава такой робот имеет гидравлический или пневматический цилиндр, который обеспечивает удлинение руки \cite{asfal:rob_auto}.
Типичным  движением робота является детали по вертикали с сохранением ее ориентации \cite{asfal:rob_auto}.
Для роботов с многозвенной рукой это представляет проблему, так как требует одновременных скоординированных движений плечевого и локтевого суставов \cite{asfal:rob_auto}.
\par
{\itshape Роботы, работающие в цилиндрических координатах}.
Робот, работающий в цилиндрических координатах имеет вертикальное возвратно--поступательное движение для второй степени подвижности или подвижное основание \cite{asfal:rob_auto}.
Обычно это выполняется с помощью пневматического или гидравлического цилиндра \cite{asfal:rob_auto}.
Роботы такого типа обычно имеют возвратно--поступательное движение по третьей степени подвижности \cite{asfal:rob_auto}.
Недостатком роботов, работающих в цилиндрических координатах, является невозможность огибать препятствия \cite{asfal:rob_auto}.
\par
{\itshape Роботы, работаюшие в декартовой системе координат}.
Рабочим, имеющим дело со станками, удобнее использовать роботы работающие в декартовой системе координат \cite{asfal:rob_auto}.
Первые три оси роботов, работающий в декартовой системе координат~--- это оси \(X,\;Y,\;Z\) металлорежущих станков \cite{asfal:rob_auto}.
Преимуществом такого типа роботов является их жесткость, обеспечиваемая за счет коробчатой конструкции рамы портала \cite{asfal:rob_auto}.

\subsection{Анализ математического аппарата}

Кинематика манипулятора изучает геометрию движения манипулятора относительно заданной абсолютной системы координат, не рассматривая силы и моменты, порождающие это движение \cite{fu:rob_tech}.
Таким образом, ее предметом является описание пространственного положения манипулятора как функции времени, и, в частности, соотношения между пространством присоединенных переменных манипулятора~--- обобщенными координатами, положением и ориентацией схвата \cite{fu:rob_tech}.
\par
Прямая задача кинематики манипулятора формулируется следующим образом:
\par
Для конкретного манипулятора по известному вектору присоединенных углов~--- обощенных координат \(q(t) = (q_1(t),q_2(t), \cdots, q_n(t))^T\) и заданным геометрическим параметрам звеньев (\(n\)~--- число степеней свободы) определить положение и ориентацию схвата манипулятора относительно абсолютной системы координат \cite{fu:rob_tech}.
\par
Обратная задача кинематики манипулятора звучит так:
\par
При известных геометрических параметрах звеньев найти все возможные векторы присоединенных переменных манипулятора, обеспечивающие заданные положение и ориентацию схвата относительно абсолютной системы координат \cite{fu:rob_tech}.
\par
Для систематического и обощенного подхода к описанию и расположению звеньев манипулятора относительно заданной абсолютной системы координат используется матричная и векторная алгебра \cite{fu:rob_tech, shahinpur:rob_tech}.
Так как звенья манипулятора могут совершать вращательное или поступательное движение относительно абсолютной системы отсчета, для каждого звена определяется связанная система координат, оси которой параллельны осям сочленений звеньев \cite{fu:rob_tech}.
Так как манипуляционные системы представляют собой разомкнутую кинематическую цепь, состоящую из абсолютно твердых звеньев, соединенных между собой кинематическими парами пятого класса, имеющими одну степень свободы, то каждое звено такой манипуляционной системы, кроме последнего, входит в две кинематические пары \cite{krahmalev:mod_rob}.
Поэтому целесообразно с каждым звеном связать две системы координат, расположенных в центрах кинематических пар этого звена \cite{krahmalev:mod_rob}.
\par
Для описания вращательного движения связанной системы отсчета относительно абсолютной используется матрица поворота размерностью \(3 \times 3\), для представления векторов положения в пространстве применяются однородные координаты \( [x\:y\:z]^T\) \cite{fu:rob_tech, shahinpur:rob_tech}.
\par
Матрицу поворота можно определить как матрицу преобразования трехмерного вектора положения в евклидовом пространстве, переводящую его координаты из связанной системы отсчета \(OUVW\) в асболютную систему координат \(OXYZ\) \cite{fu:rob_tech}.
Пусть \((i_x,\:j_y,\:k_z)\) и \((i_u,\:j_v,\:k_w)\)~--- единичные векторы, направленные вдоль осей систем \(OXYZ\) и \(OUVW\) соответственно.
Предположим, что есть некоторая точка \(p\) фиксированная и неподвижная в системе отсчета \(OUVW\).
Тогда в системах координат \(OUVW\) и \(OXYZ\) точка \(p\) будет иметь соответственно координаты \cite{fu:rob_tech}
\begin{equation}\label{f:p_uvw:c}
  p_{uvw} = (p_u,\:p_v,\:p_w)^T
  ,
\end{equation}
\begin{equation}\label{f:p_xyz:c}
  p_{xyz} = (p_x,\:p_y,\:p_z)^T
  .
\end{equation}
\par
Задача состоит в определении матрицы \(R\), которая преобразует координаты \(p_{uvw}\) в координаты вектора \(p\) в системе \(OXYZ\) после того, как система \(OUVW\) будет повернута, т. е.
\begin{equation}\label{f:p_xyz:ouvw:sh}
  p_{xyz} = R p_{uvw}
  .
\end{equation}
\par
Из определения компонент вектора имеем 
\begin{equation}\label{f:p_uvw:v}
  p_{uvw} = p_u \cdot i_u + p_v \cdot j_u + p_w \cdot k_w
  .
\end{equation}
\par
Таким образом, используя определение скалярного произведения и равенство \ref{f:p_uvw:v}, получаем 
\begin{equation}\label{f:p_xyz:ouvw}
  \begin{aligned}
  & p_x = i_x \cdot p =  i_x \cdot i_u \cdot p_u + i_x \cdot i_v \cdot p_v + i_x \cdot i_w \cdot p_w,\\
  & p_y = i_y \cdot p =  i_y \cdot i_u \cdot p_u + i_y \cdot i_v \cdot p_v + i_y \cdot i_w \cdot p_w,\\
  & p_z = i_z \cdot p =  i_z \cdot i_u \cdot p_u + i_z \cdot i_v \cdot p_v + i_z \cdot i_w \cdot p_w,
  \end{aligned}
\end{equation}
или в матричной форме
\begin{equation}\label{f:p_xyz:ouvw:m}
  \begin{bmatrix}
  p_x\\
  p_y\\
  p_z
  \end{bmatrix}
  =
  \begin{bmatrix}
  i_x \cdot i_u & i_x \cdot i_v & i_x \cdot i_w\\
  i_y \cdot i_u & i_y \cdot i_v & i_y \cdot i_w\\
  i_z \cdot i_u & i_z \cdot i_v & i_z \cdot i_w
  \end{bmatrix}
  \cdot
\begin{bmatrix}
  p_u\\
  p_v\\
  p_w
  \end{bmatrix}
  .
\end{equation}
\par
С учетом этого выражения матрица \(R\) в равенстве \ref{f:p_xyz:ouvw:sh} примет вид
\begin{equation}\label{f:r}
  R = 
  \begin{bmatrix}
  i_x \cdot i_u & i_x \cdot i_v & i_x \cdot i_w\\
  i_y \cdot i_u & i_y \cdot i_v & i_y \cdot i_w\\
  i_z \cdot i_u & i_z \cdot i_v & i_z \cdot i_w
  \end{bmatrix}
  .
\end{equation}
\par
Преобразование, определяемое формулой \ref{f:p_xyz:ouvw:sh} называется ортогональным преобразованием, а поскольку все векторы, входящие в скалярное произведение, единичные, его также называют ортонормальным преобразованием \cite{fu:rob_tech}.
\par
Если положение системы \(OUVW\) в пространстве изменяется за счет поворота этой системы на угол \(\alpha\) вокруг оси \(OX\), то матрица \(R\) примет вид
\begin{equation}\label{f:r_xalpha}
  R_{x,\alpha} = 
  \begin{bmatrix}
  1 & 0 & 0\\
  0 & \cos{\alpha} & -\sin{\alpha}\\
  0 & \sin{\alpha} & \cos{\alpha}
  \end{bmatrix}
  .
\end{equation}
\par
Аналогично матрицы поворота вокруг оси \(OY\) на угол \(\varphi\) (см. формулу \ref{f:r_yphi}) и вокруг оси \(OZ\) на угол \(\theta\) (см. формулу \ref{f:r_ztheta}).
\begin{equation}\label{f:r_yphi}
  R_{y,\varphi} =
  \begin{bmatrix}
  \cos{\varphi} & 0 & \sin{\varphi}\\
  0 & 1 & 0\\
  -\sin{\varphi} & 0 & \cos{\varphi}
  \end{bmatrix}
\end{equation}
\begin{equation}\label{f:r_ztheta}
  R_{z,\theta} =
  \begin{bmatrix}
  \cos{\theta} & -\sin{\theta} & 0\\
  \sin{\theta} & \cos{\theta} & 0\\
  0 & 0 & 1
  \end{bmatrix}
\end{equation}
\par
Матрицы \(R_{x,\alpha}\), \(R_{y,\varphi}\) и \(R_{z,\theta}\) называются матрицами элементарных поворотов \cite{fu:rob_tech}.
Любые другие матрицы конечных поворотов можно получить, используя матрицы элементарных поворотов \cite{fu:rob_tech}.
Поскольку операция перемножения матриц некоммутативна, здесь существенна последовательность выполнения поворотов \cite{fu:rob_tech}.
Результирующая матрица поворота может быть получена с помощью следующих правил \cite{fu:rob_tech}\,:
\begin{enumerate}
  \item Вначале обе системы совпадают, и, следовательно, матрица поворота представляет собой единичную матрицу \(3 \times 3\)\,;
  \item Если подвижная система координат \(OUVW\) совершает поворот вокруг одной из основных осей системы \(OXYZ\), матрицу предыдущего результирующего поворота умножают слева на соответствующую матрицу элементарного поворота;
  \item Если подвижная система координат \(OUVW\) совершает поворот вокруг одной из своих основных осей, матрицу предыдущего результирующего поворота надо умножить справа на соответствующую матрицу элементарного поворота.
\end{enumerate}
\par
Поскольку трехмерная матрица поворота не несет информации о поступательном перемещении и масштабе, вектор координат \(p = (p_x,\:p_y,\:p_z)^T\) в трехмерном пространстве дополняют четвертой компонентой так, что она принимает вид \(\widehat{p} = (\omega p_x,\:\omega p_y,\:\omega p_z,\:\omega)^T\) \cite{fu:rob_tech}.
В этом случае говорят, что вектор \(\widehat{p}\) выражен в однородных координатах \cite{fu:rob_tech}.
Четвертую компоненту \(\omega\) можно рассматривать как масштабирующий множитель, принимающий значение \(\omega > 0\).
Если \(\omega = 1\), то однородные координаты вектора положения совпадают с его физическими координатами \cite{fu:rob_tech}.
\par
{\itshape Однородная матрица преобразования} представляет собой матрицу размерностью \(4 \times 4\), которая преобразует вектор выраженный в однородных координатах, из одной системы отсчета в другую \cite{fu:rob_tech}.
Однородная матрица преобразования может быть разбита на четрые подматрицы:
\begin{displaymath}
  T =
  \begin{bmatrix}
  R_{3 \times 3} & p_{3 \times 1}\\
  f_{1 \times 3} & 1 \times 1
  \end{bmatrix}
  =
  \begin{bmatrix}
  \text{\ttfamily Поворот} & \text{\ttfamily Сдвиг}\\
  \text{\ttfamily Преобразование перспективы} & \text{\ttfamily Машстабирование}
  \end{bmatrix}
\end{displaymath}
Таким образом, однородное представление для матриц \ref{f:r_xalpha} -- \ref{f:r_ztheta} имеет соответственно вид
\begin{equation}\label{f:t_xalpha}
  T_{x,\alpha} = 
  \begin{bmatrix}
  1 & 0 & 0 & 0\\
  0 & \cos{\alpha} & -\sin{\alpha} & 0\\
  0 & \sin{\alpha} & \cos{\alpha} & 0\\
  0 & 0 & 0 & 1
  \end{bmatrix}
  ,
\end{equation}
\begin{equation}\label{f:t_yphi}
  T_{y,\varphi} = 
  \begin{bmatrix}
  \cos{\varphi} & 0 & \sin{\varphi} & 0\\
  0 & 1 & 0 & 0\\
  -\sin{\varphi} & 0 & \cos{\varphi} & 0\\
  0 & 0 & 0 & 1
  \end{bmatrix}
  ,
\end{equation}
\begin{equation}\label{f:t_ztheta}
  T_{z,\theta} = 
  \begin{bmatrix}
  \cos{\theta} & -\sin{\theta} & 0 & 0\\
  \sin{\theta} & \cos{\theta} & 0 & 0\\
  0 & 0 & 1 & 0\\
  0 & 0 & 0 & 1
  \end{bmatrix}
  .
\end{equation}
\par
Однородная матрица композиции преобразований может быть получена путем перемножения этих матриц \cite{fu:rob_tech}.
Однако, поскольку операция перемножения матриц некоммутативна, особое внимание следут обратить на порядок перемножения матриц.
При определении однородной матрицы композиции используются следующие правила \cite{fu:rob_tech}\,:
\begin{enumerate}
  \item Вначале обе системы координат совпадают, и, следовательно, однородная матрица преобразования представляет собой единичную матрицу размерностью \(4 \times 4\);
  \item Если подвижная система координат \(OUVW\) совершает поворот\,/\,сдвиг относительно осей системы отсчета \(OXYZ\), однородную матрицу предыдущего результирующего преобразования надо умножить слева на соответствующую однородную матрицу элементарного поворота\,/\,сдвига;
  \item Если подвижная система координат \(OUVW\) совершает поворот\,/\,сдвиг относительно одной из собственных основных осей, однородную матрицу предыдущего результирующего преобразования надо умножить справа на соответствующую однородную матрицу элементарного поворота\,/\,сдвига.
\end{enumerate}

\newpage

\section{Построение математической модели}


\newpage

\section{Разработка ПО}


\newpage

\section*{Заключение}
\addcontentsline{toc}{section}{Заключение}


\newpage

\addcontentsline{toc}{section}{Список литературы}
\begin{thebibliography}{00}

  \bibitem{krahmalev:mod_rob} Крахмалев, О. Н.
  \emph{Моделирование мапипуляционных систем роботов.}
  / О.\,Н.~Крахмалев.
  ~--- Саратов\,: Ай Пи Эр Медиа, 2018.~--- 165\,с.~--- Текст: непосредственный.

  \bibitem{borisov:mod_rob} Борисов, О. И.
  \emph{Методы управления робототехническими приложениями.}
  / О.\,И.~Борисов, В.\,С.~Громов, А.\,А.~Пыркин.
  ~--- СПб\,: изд-во ИТМО, 2016.~--- 108\,с.~--- Текст: непосредственый.

  \bibitem{fu:rob_tech} Фу, К.
  \emph{Робототехника.}
  / К.~Фу, Р.~Гонсалес, К.~Ли.
  ~--- пер. с анг.~--- М.\,: Мир, 1989.~--- 624\,c.,~ил.~--- Текст: непосредственный.
  
  \bibitem{shahinpur:rob_tech} Шахинпур, М.
  \emph{Курс робототехники.}
  / М. Шахинпур.
  ~--- пер. с анг.~--- М.\,: Мир, 1990.~--- 527\,с.,~ил.~--- Текст: непосредственный.

  \bibitem{asfal:rob_auto} Асфаль, Р.
  \emph{Роботы и автоматизация производства.}
  / Р. Асфаль.
  ~--- пер. с англ.~--- М.\,: Машиностроение, 1989.~--- 448\,c.,~ил.~--- Текст: непосредственный.

  \bibitem{tolstosheev:teo_meh_mash} Толстошеев, А. К.
  \emph{Теория механизмов и машин: курсовая работа в примерах.}
  / А.\,К.~Толстошеев. 
  ~--- Брянск\,: из-во БГТУ, 2019.~--- 164\,с.~--- Текст: непосредственный.

  \bibitem{kerningan:c} Кернинган, Б. У.
  \emph{Язык программирования ANSI C.}
  / Б.\,У.~Кернинган, Д.\,М.~Ритчи.
  ~--- пер. с англ.~--- СПб\,: ООО "<Диалектика">, 2020~--- 228\,с.~--- Текст: непосредственный.

\end{thebibliography}
\newpage

\appendix

\section{Программа}

% конец документа
\end{document}
